\documentclass[11pt]{article}
\usepackage{amssymb}
\usepackage{url}
\usepackage{graphicx}

%Gummi|065|=)
\title{\textbf{How can Exponential Diophantine Equations be Solved to Generate Solutions to the Euler Brick Problem?}}
\author{Ibrahim Khan}
\date{}
\begin{document}

\maketitle
\newpage
\tableofcontents
\newpage
\section{Introduction}
The Euler Brick problem, named in memory of Leonard Euler, is a problem concerning the existence of a cuboid with integer side lengths as well as face diagonals. Such cuboids can be generated by finding solutions to a system of homogenous Diophantine equations of degree two, and mathematicians over the centuries have persisted in finding the smallest such solution, with attempts by Euler and Saunderson through parameterization being particularly successful, although not entirely comprehensive. While Euler bricks exist and have been discovered, the existence of a perfect Euler brick, or one with an integer internal diagonal in addition to the requirements put forth by a normal Euler brick, has yet to be confirmed. This essay will offer an outline into general methods for the generation of Euler bricks, focusing on numerical approaches including various parameterisations, as well as potential graphical approaches. An investigation into special edge cases as part of the wider Perfect Cuboid problem will also be taken, as well as an exploration of the problems concerning and progress made with regards to discovering perfect Euler bricks.


\section{Outlining the problem}
Consider a regular cuboid with sides $x$, $y$, and $z$:

[I have yet to make a diagram specific to these variables, so a similar one has been used as a placeholder]

Introducing the three unique face diagonals $d_1$, $d_2$, and $d_3$ yields:
\begin{figure}[h]
\includegraphics[scale=0.5]{brick}
\end{figure}

These diagonals can then be related to their respective sides through Pythagoras' theorem, which gives the square of each diagonal as the sum of the squares of the sides, or "legs", of each triangle formed. Applying this to all three faces, the following system of equations is derived:
$$x^2+y^2=d_1^2$$
$$x^2+z^2=d_2^2$$
$$y^2+z^2=d_3^2$$
However, in order for the cuboid to be an Euler brick, $(x, y, z)$ and $(d_1, d_2, d_3)$ must all be such that 
$$(x, y, z)\cup(d_1, d_2, d_3)\subseteq{\mathbb{N}}$$
implying that all sides and diagonals are positive integers.
As a result, an Euler brick must have dimensions or side lengths $x$, $y$, and $z$ such that they fulfill the system of equations given above, within the subset of natural numbers. This system of equations is referred to as a Diophantine problem, or one where there are fewer equations than unknowns, in this case 3 equations and six variables, and positive integer solutions must be found by solving all three equations simultaneously. In addition to this, the fact that it involves Pythagorean triples means that each equation is exponential with a degree of two.

While linear Diophantine equations are relatively straightforward in that solutions can be found through applying modular arithmetic, many exponential Diophantine problems require parameterization to yield solutions, an approach that shall be discussed later. Firstly, however, it is important to conduct some preliminary analysis on the brick, helping form specific cases that will be useful further on.


\section{Preliminary analysis and numerical approaches}
One way of gaining a sense of the kinds of numbers that can be part of a potential Euler brick would be to analyse Pythagorean triples themselves, seeing as they form the basis of the system of equations that we aim to solve.

A Pythagorean triple of integers $a$, $b$, and $c$ is considered to be $primitive$ if $gcd(a, b, c)=1$, in other words, $a$, $b$, and $c$ all have a highest common factor of 1, also referred to as being coprime. Therefore, an infinite number of Pythagorean triples and resultingly Euler Bricks may be generated by simply multiplying a primitive solution by any positive integer $n$. (WRONG. THIS ONLY PROVES THAT INFINITE TRIPLES EXIST, NOT BRICKS.)

Looking at the first few primitive Pythagorean triples generated by a computer script:
$$3^2+4^2=5^2$$
$$5^2+12^2=13^2$$
$$8^2+15^2=17^2$$
$$7^2+24^2=25^2$$
$$21^2+20^2=29^2$$
A few observations can be made, namely:
\begin{itemize}
  \item A primitive triple may not possess more than one even edge. Since this single even edge can only be on the left hand side of the triple, this essentially implies that $c$ must not be even, therefore being odd. (PROVE)
  \item One edge is always prime. 
  \item One edge must always divide by 2, 3, or 4.
\end{itemize}
Resultingly, in order for a certain triple to be primitive, the following conditions must apply:
\begin{itemize}
  \item One of $a$ or $b$ must be even, while the other shall be odd.
  \item $c$ must be odd.
  \item $a$, $b$, and $c$ all must be coprime (PROVE)
\end{itemize}
Utilizing these three conditions, a general parameterization can be found for primitive Pythagorean triples. A parameterization allows for all variables in an implicitly defined function, that is, a function where the dependent variable is not expressed directly in terms of the independent variable(s), to be expressed in terms of a set of common variables. As a result, the parameterized function can be viewed as essentially equivalent to the original function, with the added benefit of allowing certainty in that any values of the new variable used for the parameterized equation will yield solutions that fulfill the original. In the case of Pythagorean triples, this essentially entails finding a method of defining all three variables $a$, $b$, and $c$ under a set of different common variables, something which shall now be attempted through utilizing basic algebraic methods.

Rearranging $a^2+b^2=c^2$, we get
$$a^2=c^2-b^2$$
which can then be rewritten as 
$$a^2=(c-b)(c+b)$$
where we consider $a$ to be the odd side out of $a$ and $b$.
Accordingly, this implies that both $(c-b)$ and $(c+b)$ must be square numbers themselves in order for their square root, or $a$, to be a whole number. This is confirmed by examining any random triple, for example, $(8, 15, 17)$:
$$(c-b)=(17-8)=9=3^2$$
$$(c+b)=(17+8)=25=5^2$$
$c-b$ can therefore be expressed as $n^2$, where $n\subseteq{\mathbb{N}}$.
Likewise, $c+b$ can be written as $m^2$, implying that $a^2=m^2n^2$ and $a=mn$. It must be noted that both $m$ and $n$ are odd owing to the fact that they are derived from the square roots of $c-b$ and $c+b$, which both yield odd numbers with correspondingly odd square roots. (PROVE)

Having expressed $a$ in terms of $n$ and $m$, the same can be done for $b$ and $c$. Observe that
$$(c-b)+(c+b)=2c$$
$$m^2+n^2=2c$$
$$\therefore c=\frac{m^2+n^2}{2}$$
Likewise for $b$
$$(c+b)-(c+b)=2b$$
$$m^2-n^2=2b$$
$$\therefore b=\frac{m^2-n^2}{2}$$
This formula is referred to as Euclid's formula, and allows for the generation of most Pythagorean triples \cite{tripleparam}. An attempt can be made to discover primitive Euler bricks using only the methods outlined above.

Firstly, it is observed that, given a primitive Pythagorean triple $(x, y, d_1)$, in order for said triple to be part of a solution to the Diophantine equations concerning an Euler brick, $x$ and $y$ must both be part of two separate, unique triples. Consider again that
$$x^2=(d_1-y)(d_1+y)$$
Since $d_1-y$ and $d_1+y$ can essentially be viewed as two complementary factors of $x^2$, it follows that using varying pairs of values for either will yield different values for the other two legs of the triple, and therefore multiple Pythagorean triples containing a common edge $x$. Taking $x=85$ as an example, $x^2$ can be factorized as $1^2\cdot{5}^2\cdot{17}^2$. If we take the value of $d_1+y$ to be 25, then
$$85^2=25(d_1-y)$$
and $$d_1-y=289$$
Adding and subtracting $d_1-y$ and $d_1+y$ will then yield a value for $d_1$ and $y$, in this case $157$ and $132$, respectively. This gives a Pythagorean triple of $(85, 132, 157)$, which in the case of our potential Euler brick fulfills the first equation, $x^2+y^2=d_1^2$. As a result, what must now be found is another Pythagorean triple containing the value of $x$, or $85$, on the left-hand-side. This can be achieved by using the same approach as outlined above for the first triple, but with a different pair of factors. In this case, we can take $d_2+z$ to be $17\cdot{17}\cdot{5}$. While this value of $d_2+z$ may appear to conflict with our earlier requirement for both $d_2+z$ and $d_2-z$ having to be square numbers when deriving Euclid's formula, that requirement was only necessary for the process of derivation itself, whereas here the complementing value of $d_2-z$ as $5$ ensures that the product is a square as well regardless. Ultimately, using this pair of factors gets us a value of $725$ for $c$ and $720$ for $b$, making a triple of $(85, 720, 725)$.

Up till now, our original system of Diophantine equations
$$x^2+y^2=d_1^2$$
$$x^2+z^2=d_2^2$$
$$y^2+z^2=d_3^2$$
is looking to be
$$85^2+132^2=157^2$$
$$85^2+720^2=725^2$$
$$y^2+z^2=d_3^2$$
As it turns out, having found two triples with a common edge, there really isn't any choice with regards to the remaining triple, with the values of $y$ and $z$ having already been determined by the other two triples, in this case $132$ and $72$ respectively. An interesting consequence of this is that it implies that having an Euler brick that forms three primitive Pythagorean triples is impossible, as there must always be two even edges that form part of their own triple together, or in this case, the second equation. Another key thing to note is that, with this method, there lies some uncertainty in whether the values for $y$ and $z$ derived for the system of equations will actually fulfill a triple themselves. In the case of $a=85$, this happens to be the case as $132^2+720^2=732^2$ and we end up with an Euler brick of dimensions $(85, 132, 720)$ and diagonals $(157, 732, 725)$. However, the fact that a starting value of $85$ happens to fulfill this criterion is happenstance and a result of trial and error as opposed to a methodical process that distinguishes between values that will result in an Euler brick and those which won't. For example, while taking $x$ to be $15$ also generates two unique Pythagorean triples, the values of $y$ and $z$ produced by either don't sum up to form a triple themselves. 

Having established some of the defining qualities of an Euler brick as well as discovered one through elementary algebraic methods, it must now be seen whether an entirely foolproof process can be found for generating multiple bricks.
\section{Saunderson's method and other parameterizations}
One way of ensuring that Euler bricks are generated is by attempting to formalize the approach employed in the preceding section. As it turns out, this has already been done before with a parameterization having been developed by the blind mathematician Nicholas Saunderson in the 16th century \cite[p. 429-431]{saunderson}. Instead of relying upon partial trial and error or brute-force as in the preceding section, using Saunderson's parameterisation will allow us to ascertain whether a certain number will yield an Euler brick when used as a starting point.

To start off with, instead of using arbitrary values for the factors of $x^2$ as in the preceding section, we can instead let there be two pairs of factors, defined as $(mk, \frac{x^2}{mk})$ and $(nk, \frac{x^2}{nk})$, which can be used in a similar manner to fill two of the three equations. Then, using the same properties of Pythagorean triples outlined earlier, the values of $y^2$ and $z^2$, which form the basis of the last remaining equation once the other two triples have been derived, are given as:
$$y^2=\frac{m^2k^2}{4}-\frac{x^2}{2}+\frac{x^4}{4m^2k^2}$$
and 
$$z^2=\frac{n^2k^2}{4}-\frac{x^2}{2}+\frac{x^4}{4n^2k^2}$$
The third equation in the system is $y^2+z^2=d_3^2$, which can now be given by 
$$d_3^2=\frac{m^2k^2}{4}-\frac{x^2}{2}+\frac{x^4}{4m^2k^2}+\frac{n^2k^2}{4}-\frac{x^2}{2}+\frac{x^4}{4n^2k^2}$$
$$=\frac{m^2k^2}{4}+\frac{n^2k^2}{4}-x^2+\frac{x^4}{4m^2z^2}+\frac{x^4}{4n^2z^2}$$
From here, utilizing the assumption that the above expression equates to a square number, we must find a way of expressing $x$, $y$, and $z$, i.e. the edges of the cuboid, in terms of the variables $m$ and $n$.

Looking at the sum of the first two terms of the expression, it is evident that the factors of $x$ $m$ and $n$ themselves form a Pythagorean triple, allowing for the introduction of a third variable $l$, defined as $\sqrt{m^2+n^2}$. In doing so, the sum can then be rewritten as
$$\frac{m^2k^2}{4}+\frac{n^2k^2}{4}=\frac{l^2k^2}{4}$$
In a similar fashion, the last two terms can be condensed into a singular expression involving $l$:
$$\frac{x^4}{4m^2k^2}+\frac{x^4}{4n^2k^2}=\frac{1}{4}\frac{x^4m^2+x^4n^2}{m^2n^2k^2}$$
$$=\frac{x^4l^2}{4m^2n^2k^2}$$
Ultimately yielding
$$d_3^2=\frac{l^2k^2}{4}-x^2+\frac{x^4l^2}{4m^2n^2k^2}$$
One way of fulfilling this equation, or, in essence, ensuring that the right-hand-side is a square number, is to make it such that the only term left is the very first one, which is guaranteed to be a square number. This simply means taking $d_3^2$ to be $\frac{l^2k^2}{4}$:
$$\frac{l^2k^2}{4}=\frac{l^2k^2}{4}-x^2+\frac{x^4l^2}{4m^2n^2k^2}$$
$$x^2=\frac{x^4l^2}{4m^2n^2k^2}$$
Simplifying further (dividing by $x^2$ and taking the square root), 
$$1=\frac{xl}{2mnk}$$
At this point, what we've managed to do is express one of our edges, $x$, in terms of factor pairs that assume separate variables. If we do the same with $y$ and $z$, the other two edges of the brick, we'll have successfully derived a parameterization of the Euler brick. 

Recall that we started off with assigning two different factor pairs to $x^2$, namely, $(mk, \frac{x^2}{mk})$ and $(nk, \frac{x^2}{nk})$. Also recall that, in the section prior, we derived multiple Pythagorean triples from a single $a$ using Euclid's formula. To recap, if $(a, b, c)$, arbitrary variables not in any way linked to the edges of our brick, form a Pythagorean triple,
then 
$$a^2=(c+b)(c-b)$$
$$b=\frac{(c+b)-(c-b)}{2}$$
$$c=\frac{(c+b)+(c-b)}{2}$$
Since we have two factor pairs, we can have two separate forms of $a^2$, and for each form, it is the derivative value of $b$, or the shorter leg of the triangle formed by the consequent triple, that we are interested in, as opposed to the larger hypotenuse, or $c$, which would just give us one of the face diagonals of the brick as opposed to an edge.

Rearranging what we have up till now,
$$mk=\frac{xl}{2n}$$
and 
$$\frac{x^2}{mk}=2xl$$
Since $mk$ and $\frac{x^2}{mk}$ are factors that multiply to give $x^2$, they can be rewritten in the form of $(c+b)(c-b)$, and, working from this, we can then get a value of $y$, which shall be one of the edges of the brick. Taking $c+b$ to be $mk$ and $c-b$ to be $\frac{x^2}{mk}$, we get $y$ as $\frac{xm}{2}-\frac{xl}{4m}$. Similarly, for the pair of factors $nk$ and $\frac{x^2}{nk}$, the resulting value of the side $z$ is $\frac{xn}{2}-\frac{xl}{4n}$.
Through this entire process, the sides of the brick $x$, $y$, and $z$ can now be written as
$$x=x$$
$$y=\frac{xm}{2}-\frac{xl}{4m}$$
$$z=\frac{xn}{2}-\frac{xl}{4n}$$
In order to express the above only in terms of parameters $m$, $n$, and $l$, the starting value of the side $x$ must be suitably rewritten. Since $x$ is factored by all three and $x$ can be arbitrary anyhow, Saunderson chose to let $x$ equal $4lmn$, with the coefficient of 4 allowing for cancellation with any fractions.

As a result, the sides end up being:
$$x=4lmn$$
$$y=\frac{4lmn\cdot{m}}{2}-\frac{4lmn\cdot{l}}{4m}=2lnm^2-nl^2$$
$$z=\frac{4lmn\cdot{n}}{2}-\frac{4lmn\cdot{l}}{4n}=2lmn^2-ml^2$$

Another unique parameterisation was one derived by Euler himself, however, this one differs in that it can be reduced to an elliptic curve.

Euler began by expressing the ratio of the sides $y$ and $z$ with regards to $x$ in terms of parameters $m$ and $n$. While not explicitly mentioned in \cite{euler}, these parameters are given in a manner similar to the previous approaches, as factors of $x$.

If $x^2$ has factors $mx$ and $nx$, then, applying the principles established beforehand, for an Euler brick, the other edges $y$ and $z$ are given as:
$$y=\frac{mx-\frac{x}{m}}{2}, z=\frac{nx-\frac{x}{n}}{2}$$
Therefore, 
$$\frac{y}{x}=\frac{\frac{1}{2}x(m-\frac{1}{m})}{x}=\frac{m^2-1}{2m}$$
and 
$$\frac{z}{x}=\frac{\frac{1}{2}x(n-\frac{1}{n})}{x}=\frac{n^2-1}{2n}$$
Having done so, it is now a matter of using these expressions in conjunction with the system of equations to find an expression for all three sides. Here again, in order to resolve the problem of finding resulting values of $y$ and $z$ that form a triple, the third equation from the system is used. Normally, it would be alright to continue by denoting the right-hand-side as $d_3$ as in the original equation. However, since we will be manipulating both sides of the equation while also taking care to ensure that the $rhs$ remains a square, it would be better to denote it using a square symbol from here onwards to not cause any confusion with the separate, potential value of $d_3$ itself.
$$y^2+z^2=\square$$
$$\left(\frac{y}{x}\right)^2+\left(\frac{z}{x}\right)^2=\square$$
and so
$$\left(\frac{m^2-1}{2m}\right)^2+\left(\frac{n^2-1}{2n}\right)^2=\frac{(m^2-1)^2}{4m^2}+\frac{(n^2-1)^2}{4n^2}=\square$$
Multiplying throughout by $4m^2n^2$,
$$n^2(m^2-1)^2+m^2(n^2-1)^2=\square$$
$$n^2(m-1)^2(m+1)^2+m^2(n-1)^2(n+1)^2=\square$$
The above expression may be simplified by finding a way of dividing it entirely by $(m+1)^2$ and expressing it in terms of $m$ only. Euler did this by letting $n-1$ equal $m+1$ so that the expression ends up being
$$(m+2)^2(m+1)^2(m-1)^2+m^2(m+3)^2(m+1)^2=\square$$
Which expands and simplifies to give
$$2m^4+8m^3+6m^2-4m+4=\square$$
Since the constant term of this polynomial is 4 and it equates to a square number, it is safe to assume that it can be factorised as
$$(Am^2+Bm+2)^2$$
which expands to give
$$A^2m^4+2ABm^3+4Am^2+B^2m^2+4Bm+4$$
\section{Potential graphical approaches}
\section{The existence of a perfect Euler Brick}
\section{Near-perfect Euler Bricks and their generation}
\section{Euler Bricks in higher dimensions}
\section{Applicability and potential use cases}
\section{Conclusion}
\newpage
\bibliographystyle{ieeetr}
\bibliography{citation} 
\end{document}
