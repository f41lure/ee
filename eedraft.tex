\documentclass[11pt]{article}
\usepackage{amssymb}
\usepackage{url}
\usepackage{graphicx}

%Gummi|065|=)
\title{\textbf{What Techniques Have Mathematicians Used in Attempting to Solve the Rational Cuboid Problem?}}
\author{Ibrahim Khan}
\date{}
\begin{document}

\maketitle
\newpage
\tableofcontents
\newpage
\section{Introduction}
The Rational Cuboid Problem is an umbrella term used to refer to either partially or wholly perfect cuboids, such as the Euler Brick and the Perfect Cuboid. The Euler Brick, named in memory of Leonard Euler, is a rectangular parallelepiped with integer dimensions and face diagonals, while the Perfect Cuboid is an Euler Brick that also fulfills the additional criterion of having an integer body diagonal. Both these problems can be summarised as sets of three and four quadratic Diophantine equations in six and seven variables respectively. While the Euler Brick problem has been largely solved through elementary algebraic techniques, no such method has been found for generating Perfect Cuboids, and indeed the existence of a Perfect Cuboid still remains a source of ongoing debate. 

This essay will provide an account of the variety of techniques that have been used to approach both problems. For the Euler Brick problem, this will entail an outline of the parameterisations discovered by Nicholas Saunderson and Leonard Euler himself, as well as derivative discoveries made thereafter. Prior to that, however, an explanation of the general method used in both parameterisations will be offered by way of an example. The latter half of the essay will predominantly focus on the Perfect Cuboid problem, first detailing problems that have shown to be equivalent through basic Euclidean geometry as well as demonstrating the existence of perfect parallepiped, before giving a glimpse into the role of algebraic geometry in finding possible solutions, a field of mathematics that is often employed when solving Diophantine problems.

\section{The Euler Brick Problem}
Consider a regular cuboid with sides $x$, $y$, and $z$:

[I have yet to make a diagram specific to these variables, so a similar one has been used as a placeholder]

Introducing the three unique face diagonals $d_1$, $d_2$, and $d_3$ yields:
%\begin{figure}[h]
%\includegraphics[scale=0.5]{}
%\end{figure}

These diagonals can then be related to their respective sides through Pythagoras' theorem, which gives the square of each diagonal as the sum of the squares of the sides, or "legs", of each triangle formed. Applying this to all three faces, the following system of equations is derived:
$$x^2+y^2=d_1^2$$
$$x^2+z^2=d_2^2$$
$$y^2+z^2=d_3^2$$
However, in order for the cuboid to be an Euler brick, $(x, y, z)$ and $(d_1, d_2, d_3)$ must all be such that 
$${\{x, y, z\}}\cup{\{d_1, d_2, d_3\}}\subseteq{\mathbb{N}}$$
implying that all sides and diagonals are positive integers.
As a result, an Euler brick must have dimensions or side lengths $x$, $y$, and $z$ such that they fulfill the system of equations given above, within the subset of natural numbers. This system of equations is referred to as a Diophantine problem, or one where there are fewer equations than unknowns, in this case 3 equations and six variables, and positive integer solutions must be found by solving all three equations simultaneously. In addition to this, the fact that it involves Pythagorean triples means that each equation is exponential with a degree of two.

While linear Diophantine equations are relatively straightforward in that solutions can be found through applying modular arithmetic, many exponential Diophantine problems require parameterization to yield solutions, an approach that shall be discussed later. Firstly, however, the underlying basis of these parameterisations must be defined.

\section{Preliminary analysis and numerical approaches}
One way of gaining a sense of the kinds of numbers that can be part of a potential Euler brick would be to analyse Pythagorean triples themselves, seeing as they form the basis of the system of equations that we aim to solve.

A Pythagorean triple of integers $a$, $b$, and $c$ is considered to be $primitive$ if $gcd(a, b, c)=1$, in other words, $a$, $b$, and $c$ all have a highest common factor of 1, also referred to as being coprime. Therefore, an infinite number of Pythagorean triples and resultingly Euler Bricks may be generated by simply multiplying a primitive solution by any positive integer $n$. (WRONG. THIS ONLY PROVES THAT INFINITE TRIPLES EXIST, NOT BRICKS.)

Looking at the first few primitive Pythagorean triples generated by a computer script:
$$3^2+4^2=5^2$$
$$5^2+12^2=13^2$$
$$8^2+15^2=17^2$$
$$7^2+24^2=25^2$$
$$21^2+20^2=29^2$$
A few observations can be made, namely:
\begin{itemize}
  \item A primitive triple may not possess more than one even edge. Since this single even edge can only be on the left hand side of the triple, this essentially implies that $c$ must not be even, therefore being odd. (PROVE)
  \item One edge is always prime. 
  \item One edge must always divide by 2, 3, or 4.
\end{itemize}
Resultingly, in order for a certain triple to be primitive, the following conditions must apply:
\begin{itemize}
  \item One of $a$ or $b$ must be even, while the other shall be odd.
  \item $c$ must be odd.
  \item $a$, $b$, and $c$ all must be coprime (PROVE)
\end{itemize}
Utilizing these three conditions, a general parameterization can be found for primitive Pythagorean triples. A parameterization allows for all variables in an implicitly defined function, that is, a function where the dependent variable is not expressed directly in terms of the independent variable(s), to be expressed in terms of a set of common variables. As a result, the parameterized function can be viewed as essentially equivalent to the original function, with the added benefit of allowing certainty in that any values of the new variable used for the parameterized equation will yield solutions that fulfill the original. In the case of Pythagorean triples, this essentially entails finding a method of defining all three variables $a$, $b$, and $c$ under a set of different common variables, something which shall now be attempted through utilizing basic algebraic methods.

Rearranging $a^2+b^2=c^2$, we get
$$a^2=c^2-b^2$$
which can then be rewritten as 
$$a^2=(c-b)(c+b)$$
where we consider $a$ to be the odd side out of $a$ and $b$.
Accordingly, this implies that both $(c-b)$ and $(c+b)$ must be square numbers themselves in order for their square root, or $a$, to be a whole number. This is confirmed by examining any random triple, for example, $(8, 15, 17)$:
$$(c-b)=(17-8)=9=3^2$$
$$(c+b)=(17+8)=25=5^2$$
$c-b$ can therefore be expressed as $n^2$, where $n\subseteq{\mathbb{N}}$.
Likewise, $c+b$ can be written as $m^2$, implying that $a^2=m^2n^2$ and $a=mn$. It must be noted that both $m$ and $n$ are odd owing to the fact that they are derived from the square roots of $c-b$ and $c+b$, which both yield odd numbers with correspondingly odd square roots. (PROVE)

Having expressed $a$ in terms of $n$ and $m$, the same can be done for $b$ and $c$. Observe that
$$(c-b)+(c+b)=2c$$
$$m^2+n^2=2c$$
$$\therefore c=\frac{m^2+n^2}{2}$$
Likewise for $b$
$$(c+b)-(c+b)=2b$$
$$m^2-n^2=2b$$
$$\therefore b=\frac{m^2-n^2}{2}$$
This formula is referred to as Euclid's formula, and allows for the generation of most Pythagorean triples \cite{tripleparam}. An attempt can be made to discover primitive Euler bricks using only the methods outlined above.

Firstly, it is observed that, given a primitive Pythagorean triple $(x, y, d_1)$, in order for said triple to be part of a solution to the Diophantine equations concerning an Euler brick, $x$ and $y$ must both be part of two separate, unique triples. Consider again that
$$x^2=(d_1-y)(d_1+y)$$
Since $d_1-y$ and $d_1+y$ can essentially be viewed as two complementary factors of $x^2$, it follows that using varying pairs of values for either will yield different values for the other two legs of the triple, and therefore multiple Pythagorean triples containing a common edge $x$. Taking $x=85$ as an example, $x^2$ can be factorized as $1^2\cdot{5}^2\cdot{17}^2$. If we take the value of $d_1+y$ to be 25, then
$$85^2=25(d_1-y)$$
and $$d_1-y=289$$
Adding and subtracting $d_1-y$ and $d_1+y$ will then yield a value for $d_1$ and $y$, in this case $157$ and $132$, respectively. This gives a Pythagorean triple of $(85, 132, 157)$, which in the case of our potential Euler brick fulfills the first equation, $x^2+y^2=d_1^2$. As a result, what must now be found is another Pythagorean triple containing the value of $x$, or $85$, on the left-hand-side. This can be achieved by using the same approach as outlined above for the first triple, but with a different pair of factors. In this case, we can take $d_2+z$ to be $17\cdot{17}\cdot{5}$. While this value of $d_2+z$ may appear to conflict with our earlier requirement for both $d_2+z$ and $d_2-z$ having to be square numbers when deriving Euclid's formula, that requirement was only necessary for the process of derivation itself, whereas here the complementing value of $d_2-z$ as $5$ ensures that the product is a square as well regardless. Ultimately, using this pair of factors gets us a value of $725$ for $c$ and $720$ for $b$, making a triple of $(85, 720, 725)$.

Up till now, our original system of Diophantine equations
$$x^2+y^2=d_1^2$$
$$x^2+z^2=d_2^2$$
$$y^2+z^2=d_3^2$$
is looking to be
$$85^2+132^2=157^2$$
$$85^2+720^2=725^2$$
$$y^2+z^2=d_3^2$$
As it turns out, having found two triples with a common edge, there really isn't any choice with regards to the remaining triple, with the values of $y$ and $z$ having already been determined by the other two triples, in this case $132$ and $72$ respectively. An interesting consequence of this is that it implies that having an Euler brick that forms three primitive Pythagorean triples is impossible, as there must always be two even edges that form part of their own triple together, or in this case, the second equation. Another key thing to note is that, with this method, there lies some uncertainty in whether the values for $y$ and $z$ derived for the system of equations will actually fulfill a triple themselves. In the case of $a=85$, this happens to be the case as $132^2+720^2=732^2$ and we end up with an Euler brick of dimensions $(85, 132, 720)$ and diagonals $(157, 725, 732)$. However, the fact that a starting value of $85$ happens to fulfill this criterion is happenstance and a result of trial and error as opposed to a methodical process that distinguishes between values that will result in an Euler brick and those which won't. For example, while taking $x$ to be $15$ also generates two unique Pythagorean triples, the values of $y$ and $z$ produced by either don't sum up to form a triple themselves. 

Having established some of the defining qualities of an Euler brick as well as discovered one through elementary algebraic methods, it must now be seen whether an entirely foolproof process can be found for generating multiple bricks.
\section{Saunderson's method and other parameterizations}
One way of ensuring that Euler bricks are generated is by attempting to formalize the approach employed in the preceding section. As it turns out, this has already been done before with a parameterization having been developed by the blind mathematician Nicholas Saunderson in the 16th century \cite[p. 429-431]{saunderson}. Instead of relying upon partial trial and error or brute-force as in the preceding section, using Saunderson's parameterisation will allow us to ascertain whether a certain number will yield an Euler brick when used as a starting point.

To start off with, instead of using arbitrary values for the factors of $x^2$ as in the preceding section, we can instead let there be two pairs of factors, defined as $(mk, \frac{x^2}{mk})$ and $(nk, \frac{x^2}{nk})$, which can be used in a similar manner to fill two of the three equations. Then, using the same properties of Pythagorean triples outlined earlier, the values of $y^2$ and $z^2$, which form the basis of the last remaining equation once the other two triples have been derived, are given as:
$$y^2=\frac{m^2k^2}{4}-\frac{x^2}{2}+\frac{x^4}{4m^2k^2}$$
and 
$$z^2=\frac{n^2k^2}{4}-\frac{x^2}{2}+\frac{x^4}{4n^2k^2}$$
The third equation in the system is $y^2+z^2=d_3^2$, which can now be given by 
$$d_3^2=\frac{m^2k^2}{4}-\frac{x^2}{2}+\frac{x^4}{4m^2k^2}+\frac{n^2k^2}{4}-\frac{x^2}{2}+\frac{x^4}{4n^2k^2}$$
$$=\frac{m^2k^2}{4}+\frac{n^2k^2}{4}-x^2+\frac{x^4}{4m^2z^2}+\frac{x^4}{4n^2z^2}$$
From here, utilizing the assumption that the above expression equates to a square number, we must find a way of expressing $x$, $y$, and $z$, i.e. the edges of the cuboid, in terms of the variables $m$ and $n$.

Looking at the sum of the first two terms of the expression, it is evident that the factors of $x$ $m$ and $n$ themselves form a Pythagorean triple, allowing for the introduction of a third variable $l$, defined as $\sqrt{m^2+n^2}$. In doing so, the sum can then be rewritten as
$$\frac{m^2k^2}{4}+\frac{n^2k^2}{4}=\frac{l^2k^2}{4}$$
In a similar fashion, the last two terms can be condensed into a singular expression involving $l$:
$$\frac{x^4}{4m^2k^2}+\frac{x^4}{4n^2k^2}=\frac{1}{4}\frac{x^4m^2+x^4n^2}{m^2n^2k^2}$$
$$=\frac{x^4l^2}{4m^2n^2k^2}$$
Ultimately yielding
$$d_3^2=\frac{l^2k^2}{4}-x^2+\frac{x^4l^2}{4m^2n^2k^2}$$
One way of fulfilling this equation, or, in essence, ensuring that the right-hand-side is a square number, is to make it such that the only term left is the very first one, which is guaranteed to be a square number. This simply means taking $d_3^2$ to be $\frac{l^2k^2}{4}$:
$$\frac{l^2k^2}{4}=\frac{l^2k^2}{4}-x^2+\frac{x^4l^2}{4m^2n^2k^2}$$
$$x^2=\frac{x^4l^2}{4m^2n^2k^2}$$
Simplifying further (dividing by $x^2$ and taking the square root), 
$$1=\frac{xl}{2mnk}$$
At this point, what we've managed to do is express one of our edges, $x$, in terms of factor pairs that assume separate variables. If we do the same with $y$ and $z$, the other two edges of the brick, we'll have successfully derived a parameterization of the Euler brick. 

Recall that we started off with assigning two different factor pairs to $x^2$, namely, $(mk, \frac{x^2}{mk})$ and $(nk, \frac{x^2}{nk})$. Also recall that, in the section prior, we derived multiple Pythagorean triples from a single $a$ using Euclid's formula. To recap, if $(a, b, c)$, arbitrary variables not in any way linked to the edges of our brick, form a Pythagorean triple,
then 
$$a^2=(c+b)(c-b)$$
$$b=\frac{(c+b)-(c-b)}{2}$$
$$c=\frac{(c+b)+(c-b)}{2}$$
Since we have two factor pairs, we can have two separate forms of $a^2$, and for each form, it is the derivative value of $b$, or the shorter leg of the triangle formed by the consequent triple, that we are interested in, as opposed to the larger hypotenuse, or $c$, which would just give us one of the face diagonals of the brick as opposed to an edge.

Rearranging what we have up till now,
$$mk=\frac{xl}{2n}$$
and 
$$\frac{x^2}{mk}=2xl$$
Since $mk$ and $\frac{x^2}{mk}$ are factors that multiply to give $x^2$, they can be rewritten in the form of $(c+b)(c-b)$, and, working from this, we can then get a value of $y$, which shall be one of the edges of the brick. Taking $c+b$ to be $mk$ and $c-b$ to be $\frac{x^2}{mk}$, we get $y$ as $\frac{xl}{4m}-xl$. Similarly, for the pair of factors $nk$ and $\frac{x^2}{nk}$, the resulting value of the side $z$ is $\frac{xl}{4n}-xl$.
Through this entire process, the sides of the brick $x$, $y$, and $z$ can now be written as
$$x=x$$
$$y=\frac{xl}{4m}-xl$$
$$z=\frac{xl}{4n}-xl$$
In order to express the above only in terms of parameters $m$, $n$, and $l$, the starting value of the side $x$ must be suitably rewritten. Since $x$ is factored by all three and $x$ can be arbitrary anyhow, Saunderson chose to let $x$ equal $4lmn$, with the coefficient of 4 allowing for cancellation with any fractions.

As a result, the sides end up being:
$$x=4lmn$$
$$y=\frac{l\cdot{4lmn}}{4m}-4lmn\cdot{l}=nl^2-4mnl^2$$
$$z=\frac{l\cdot{4lmn}}{4n}-4lmn\cdot{l}=ml^2-4mnl^2$$

Another unique parameterisation was one derived by Euler himself, however, this one differs in that it can be reduced to an elliptic curve.

Euler began by expressing the ratio of the sides $y$ and $z$ with regards to $x$ in terms of parameters $m$ and $n$. While not explicitly mentioned in \cite{euler}, these parameters are given in a manner similar to the previous approaches, as factors of $x$.

If $x^2$ has factors $mx$ and $nx$, then, applying the principles established beforehand, for an Euler brick, the other edges $y$ and $z$ are given as:
$$y=\frac{mx-\frac{x}{m}}{2}, z=\frac{nx-\frac{x}{n}}{2}$$
Therefore, 
$$\frac{y}{x}=\frac{\frac{1}{2}x(m-\frac{1}{m})}{x}=\frac{m^2-1}{2m}$$
and 
$$\frac{z}{x}=\frac{\frac{1}{2}x(n-\frac{1}{n})}{x}=\frac{n^2-1}{2n}$$
Having done so, it is now a matter of using these expressions in conjunction with the system of equations to find an expression for all three sides. Here again, in order to resolve the problem of finding resulting values of $y$ and $z$ that form a triple, the third equation from the system is used. Normally, it would be alright to continue by denoting the right-hand-side as $d_3$ as in the original equation. However, since we will be manipulating both sides of the equation while also taking care to ensure that the $rhs$ remains a square, it would be better to denote it using a square symbol from here onwards to not cause any confusion with the separate, potential value of $d_3$ itself.
$$y^2+z^2=\square$$
$$\left(\frac{y}{x}\right)^2+\left(\frac{z}{x}\right)^2=\square$$
and so
$$\left(\frac{m^2-1}{2m}\right)^2+\left(\frac{n^2-1}{2n}\right)^2=\frac{(m^2-1)^2}{4m^2}+\frac{(n^2-1)^2}{4n^2}=\square$$
Multiplying throughout by $4m^2n^2$,
$$n^2(m^2-1)^2+m^2(n^2-1)^2=\square$$
$$n^2(m-1)^2(m+1)^2+m^2(n-1)^2(n+1)^2=\square$$
The above expression may be simplified by finding a way of dividing it entirely by $(m+1)^2$ and expressing it in terms of $m$ only. Euler did this by letting $n-1$ equal $m+1$ so that the expression ends up being
$$(m+2)^2(m+1)^2(m-1)^2+m^2(m+3)^2(m+1)^2=\square$$
Which expands and simplifies to give
$$2m^4+8m^3+6m^2-4m+4=\square$$
Since the constant term of this polynomial is 4 and it equates to a square number, it is safe to assume that it can be factorised as
$$(Am^2+Bm+2)^2$$
which expands to give
$$A^2m^4+2ABm^3+4Am^2+B^2m^2+4Bm+4$$

\section{The Perfect Cuboid}
Up until the now, the only variation of this problem, and indeed, the only one with any known solutions, that we've considered is what is termed the Euler brick problem. However, this problem is just a variation of a larger, yet unsolved problem termed the perfect cuboid problem. 
Whereas the Euler brick only concerns the existence of a cuboid with integer edges and face diagonals, the perfect cuboid problem adds another condition; an integer body diagonal. This results in the introduction of another equation and hence another variable to the preceding system of equations for the Euler brick. This means that for a perfect cuboid with edges $x, y, z$,
$$x^2+y^2=d_1^2$$
$$x^2+z^2=d_2^2$$
$$y^2+z^2=d_3^2$$
and
$$x^2+y^2+z^2=d_4^2$$
where $d_4$ is the internal body diagonal of the cuboid. 

Unlike the Euler brick, no such parameterisations exist for the perfect cuboid, and indeed no perfect cuboids have been found through brute-force methods. However, this does not prove neither the existence nor the inexistence of a perfect cuboid, and hence this section will attempt to summarise some of the current approaches towards a possible solution. In general, most work done on the perfect cuboid is based on algebraic geometry and elliptic curves, both of which are commonly used as approaches to similar Diophantine problems. In addition to this, work has also been done with regards to changing the form of the problem itself to ones equivalent as well as investigating special cases of the problem, in particular the existence of a perfect normal parallepiped.

\section{Equivalences between the Perfect Cuboid and related problems}
\subsection{The Perfect Square Triangle Problem}
It is found that the perfect cuboid problem can be transformed into a variety of different problems, one of which is the perfect square triangle.

A perfect square triangle, while not formally defined, can be taken as a triangle with sides that are perfect squares, as well as rational angle bisectors. Florian Luca has demonstrated that a solution to the rational/perfect cuboid problem would result in a corresponding solution to a perfect square triangle, and vice versa, making the two equivalent.

As before, we let $x, y, z$ be the edges of our theoretical perfect cuboid, and let $d_1, d_2, d_3$ be the face diagonals and $d_4$ be the body diagonal. Now, let $(a, b, c)=(d_1^2, d_2^2, d_3^2)\in{\mathbb{N}^3}$ for convenience. Since the sum of either two of $a, b,$ and $c$ is more than the remaining number, by the triangle equality, a triangle with sides that are perfect squares can be formed as such:

[diagram of triangle with sides a, b, and c] 

Now we let $p$ equal the semiperimeter of the triangle, so that $p=\frac{a+b+c}{2}$. The semiperimeter is simply half the perimeter of the triangle, and will be used to calculate the lengths of the angle bisectors of the triangle later on. The semiperimeter of this triangle is then equal to the square of the face diagonal of the original, theoretical perfect cuboid, since
$$p=\frac{a+b+c}{2}=\frac{d_1^2+d_2^2+d_3^2}{2}$$
$$=\frac{x^2+y^2+x^2+z^2+y^2+z^2}{2}$$
$$=x^2+y^2+z^2$$
which, owing to the fact that $x, y, z$ are sides of a perfect cuboid, equals the square of the face diagonal, $d_4$. Hence,
$$p=d_4^2$$
Now, the link between the perfect cuboid and the perfect square triangle is constructed through the fact that the square of the semiperimeter, which as we know must be a whole number owing to $p, d_4^2$ being square, is used to calculate the lengths of the angle bisectors of the triangle. In order to avoid further confusion by introducing new free variables to set an example, we shall instead demonstrate this on our perfect square triangle instead. 

The angle bisectors, or the lines that divide each angle of the triangle equally, are given as $l_a, l_b,$ and $l_c$ for angles opposite to sides $a, b,$ and $c$ respectively. Then, the lengths of these bisectors are given as
$$l_a=\frac{\sqrt{bcp(p-a)}}{b+c}$$
$$l_b=\frac{\sqrt{acp(p-b)}}{a+c}$$
$$l_c=\frac{\sqrt{abp(p-c)}}{a+b}$$
Since all the numbers forming the product inside the root of the numerator are square, it follows that the triangle must have rational angle bisectors with a whole numerator for each as defined above.

As a result, if a perfect cuboid does exist, then so does a perfect square triangle as outlined above.

An interesting corollary of the above is that the perfect square triangle is also a Heronian triangle. A Heronian triangle is a triangle is a triangle with an integer area as well as integer sides, and since $\sqrt{p(p-a)(p-b)(p-c)}$, which gives the area of the above triangle, is a perfect square, the perfect square triangle is also Heronian. Currently, the only known Heronian triangles with square sides are triangles with the sides $(1853^2, 4380^2, 4427^2)$ and $(11789^2, 68104^2, 68595^2)$. Working back to obtain a cuboid by setting the sides to its face diagonals, it becomes apparent that neither of these is perfect square triangle. However, of interest is the fact that, for the first triangle, one of the edges is an integer with a value of $1387$, and for the second triangle, the corresponding cuboid has an integer body diagonal of $68595$. Unfortunately, for both, the rest of the edges are not whole numbers. 
\subsection{Heronian Tetrahedra}
Heronian tetrahedra are tetrahedra with sides, face areas, and volume all rational numbers. Since the former two criteria apply to the triangular faces of such tetrahedra, Heronian tetrahedra are composed of Heronian triangles. An interesting property of Heronian tetrahedra is that each example corresponds to an almost-perfect cuboid, one where all the edges, face, and body diagonals are integers bar one. In the case of Heronian tetrahedra, this is always one of the face diagonals. In order to see why this is so, we must first establish a link between Heronian tetrahedra and rational cuboids.

A trirectangular tetrahedron is one where 3 angles at one of the vertices are right angles. The trirectangular tetrahedron is directly linked with the Euler brick problem, as all trirectangular tetrahedra with the orthogonal edges integers give a corresponding Euler brick. This is a rather simple observation; the three edges pair up in triples as seen in the diagram below to give the same set of equations for the Euler brick:

[diagram]

However, the uncertainty still remains with regards to the possibility of a perfect cuboid being generated by a trirectangular tetrahedron fulfilling the above criterion. It shall now be demonstrated that for the brick to be a perfect cuboid, the tetrahedron must also be Heronian. 

The base of a trirectangular tetrahedron is the face opposite the vertex with orthogonal sides. If $(a, b, c)$ form the orthogonal sides, then the semiperimeter of the base is $\frac{1}{2}\left(\sqrt{a^2+b^2}+\sqrt{a^2+c^2}+\sqrt{b^2+c^2}\right)$. The area of the base is then $\sqrt{(ab)^2+(ac)^2+(bc)^2}$. If this area is a whole number, then that implies that, potentially, $(ab)^2+(ac)^2+(bc)^2=d_4^2$ for some $d_4^2$ that is also the body diagonal of a perfect cuboid. If this is the case, then the set of equations for a regular Euler brick must also hold for this set of edges. For conciseness, the proof that these equations do indeed hold true is attached in the appendix.

Since the area of the base is an integer as well, the base is a Heronian triangle. By 
\section{The Perfect Parallepiped}
\section{Approaches uing algebraic geometry}
\section{The Cuboid Conjectures}
\section{Conclusion}
\newpage
\bibliographystyle{ieeetr}
\bibliography{citation} 
\end{document}
